% -----------------------------------------------
% Template for SMAC SMC 2013
% adapted and corrected from the template for SMC 2012, which was adapted from that of SMC 2011
% further updated for TENOR 2015, 2016 and 2017
% -----------------------------------------------

\documentclass{article}
\usepackage{tenor2017}
\usepackage{ifpdf}
\usepackage[english]{babel}
\usepackage{balance}

%%%%%%%%%%%%%%%%%%%%%%%% Some useful packages %%%%%%%%%%%%%%%%%%%%%%%%%%%%%%%
%%%%%%%%%%%%%%%%%%%%%%%% See related documentation %%%%%%%%%%%%%%%%%%%%%%%%%%
%\usepackage{amsmath} % popular packages from Am. Math. Soc. Please use the 
%\usepackage{amssymb} % related math environments (split, subequation, cases,
%\usepackage{amsfonts}% multline, etc.)
%\usepackage{bm}      % Bold Math package, defines the command \bf{}
%\usepackage{paralist}% extended list environments
%%subfig.sty is the modern replacement for subfigure.sty. However, subfig.sty 
%%requires and automatically loads caption.sty which overrides class handling 
%%of captions. To prevent this problem, preload caption.sty with caption=false 
%\usepackage[caption=false]{caption}
%\usepackage[font=footnotesize]{subfig}


%user defined variables
\def\papertitle{INScore Web}
\def\authors{D. Fober \qquad Y. Orlarey \qquad S. Letz}
\def\firstauthor{Dominique Fober}
\def\secondauthor{Yann Orlarey}
\def\thirdauthor{Stéphane Letz}

% adds the automatic
% Saves a lot of ouptut space in PDF... after conversion with the distiller
% Delete if you cannot get PS fonts working on your system.

% pdf-tex settings: detect automatically if run by latex or pdflatex
\newif\ifpdf
\ifx\pdfoutput\relax
\else
   \ifcase\pdfoutput
      \pdffalse
   \else
      \pdftrue
\fi

\ifpdf % compiling with pdflatex
  \usepackage[pdftex,
    pdftitle={\papertitle},
    pdfauthor={\firstauthor, \secondauthor, \thirdauthor},
    bookmarksnumbered, % use section numbers with bookmarks
    pdfstartview=XYZ % start with zoom=100% instead of full screen; 
                     % especially useful if working with a big screen :-)
   ]{hyperref}
  %\pdfcompresslevel=9

  \usepackage[pdftex]{graphicx}
  % declare the path(s) where your graphic files are and their extensions so 
  %you won't have to specify these with every instance of \includegraphics
  \graphicspath{{./figures/}}
  \DeclareGraphicsExtensions{.pdf,.jpeg,.png}

  \usepackage[figure,table]{hypcap}

\else % compiling with latex
  \usepackage[dvips,
    bookmarksnumbered, % use section numbers with bookmarks
    pdfstartview=XYZ % start with zoom=100% instead of full screen
  ]{hyperref}  % hyperrefs are active in the pdf file after conversion

  \usepackage[dvips]{epsfig,graphicx}
  % declare the path(s) where your graphic files are and their extensions so 
  %you won't have to specify these with every instance of \includegraphics
  \graphicspath{{./figures/}}
  \DeclareGraphicsExtensions{.eps}

  \usepackage[figure,table]{hypcap}
\fi

%setup the hyperref package - make the links black without a surrounding frame
\hypersetup{
    colorlinks,%
    citecolor=black,%
    filecolor=black,%
    linkcolor=black,%
    urlcolor=black
}


% Title.
% ------
\title{\papertitle}

% Authors
% Please note that submissions are NOT anonymous, therefore 
% authors' names have to be VISIBLE in your manuscript. 
%
% Single address
% To use with only one author or several with the same address
% ---------------
\oneauthor
   {\authors} {Grame \\ %
  Centre national de création musicale \\
  Lyon - France \\
     {\tt \href{mailto:fober@grame.fr}{fober@grame.fr}}}

%Two addresses
%--------------
% \twoauthors
%   {\firstauthor} {Affiliation1 \\ %
%     {\tt \href{mailto:author1@adomain.org}{author1@adomain.org}}}
%   {\secondauthor} {Affiliation2 \\ %
%     {\tt \href{mailto:author2@adomain.org}{author2@adomain.org}}}

% Three addresses
% --------------
% \threeauthors
%   {\firstauthor} {Affiliation1 \\ %
%     {\tt \href{mailto:author1@adomain.org}{author1@adomain.org}}}
%   {\secondauthor} {Affiliation2 \\ %
%     {\tt \href{mailto:author2@adomain.org}{author2@adomain.org}}}
%   {\thirdauthor} { Affiliation3 \\ %
%     {\tt \href{mailto:author3@adomain.org}{author3@adomain.org}}}


% ***************************************** the document starts here ***************
\begin{document}
%
\capstartfalse
\maketitle
\capstarttrue
%
\begin{abstract}
Web applications become more and more popular. The ubiquity of web browsers, the advantage of always up-to-date web applications and the inherent support for cross-platform compatibility are among the key reason for their popularity. The music domain is not immune to this trend. Online musical applications are available, including in the music notation domain. Although most of the approaches are based on a classical music engraving approach, or take advantage of the natural sharing facilities of the Internet, the new Web dimension opens the door to innovative musical practices. Indeed, if distributed music scores and collaborative design are among the new features provided by the Internet, the network dimension could also impact the concept of static music score. This paper will address the technological aspects of the Web deployment of music applications and their implication on the music notation and representation, with a focus on the latest developments of INScore, an environment for the design of augmented, interactive music scores.
\end{abstract}
%

\section{Introduction}\label{sec:introduction}
This template includes all the information about formatting manuscripts for 
the TENOR 2017 Conference.
Please use \LaTeX{} templates when 
preparing your submission.
Please follow these guidelines to give the final proceedings a professional look.
If you have any questions, please contact the TENOR 2017 Organizers.
This template can be downloaded from:\\
\url{http://tenor2017.tenor-conference.org}.


\section{Conclusions}
To finish your full-length paper, end it with a conclusion;
and after careful editing and a final spell-cheek,
submit it through the \href{https://easychair.org/conferences/?conf=tenor2017}{\textcolor {magenta} {\underline {EasyChair Submission System}}}. 
\underline{Do not} send papers directly by e-mail.
%
\begin{acknowledgments}
You may acknowledge people, projects, 
funding agencies, etc. 
which can be included after the second-level heading
``Acknowledgments'' (with no numbering).
\end{acknowledgments} 

%%%%%%%%%%%%%%%%%%%%%%%%%%%%%%%%%%%%%%%%%%%%%%%%%%%%%%%%%%%%%%%%%%%%%%%%%%%%%
%bibliography here
\balance
\bibliography{../interlude}

\end{document}
